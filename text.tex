\documentclass{template/template}

%\renewcommand{\familydefault}{\sfdefault} %% Only if the base font of the document is to be sans serif
%\usepackage{bera}

\usepackage[T1]{fontenc} % evropské uvozovky
\usepackage{subcaption}
\usepackage{amsmath}
\usepackage{enumitem}
\usepackage{hyperref}
\usepackage{gensymb} % balíček symbolů
\usepackage{booktabs}
%\usepackage{lmodern}
\usepackage{csquotes} % text lze uvést do uvozovek pomocí \enquote{text}
\usepackage{textcomp}

\usepackage[toc,page]{appendix}
\usepackage{color} % balíček pro obarvování textů
\usepackage{xcolor}  % zapne možnost používání barev, mj. pro \definecolor
\definecolor{mygreen}{RGB}{0,153,153} % nastavení barev odkazů 
\definecolor{myblue}{RGB}{0,0,200} 
\definecolor{commentgreen}{RGB}{0,100,0} % nastavení barev pro příklady z C++
\definecolor{deepblue}{rgb}{0,0,0.7}
\definecolor{deepred}{rgb}{0.6,0,0}
\definecolor{deepgreen}{rgb}{0,0.5,0}
\usepackage{listings} % balíček pro formátování zdrojových kódů 
\usepackage[author=,status=draft]{fixme} % vkládání poznámek  
% dva módy (status): draft (poznámky se zobrazují v PDF) / final (poznámky se nezobrazují v PDF)
\usepackage{graphicx}
\usepackage{multirow}
\usepackage{float}


\usepackage[T1]{fontenc} % import tučného písma typu tt pro prostředí listings
%\usepackage{listings} % balíček pro obarvování syntaxe ukázek programů v textu
\usepackage{listingsutf8} %nutné pro \usepackage{listings} aby to jelo v UTF-8
\lstset{
	extendedchars 	= false,
	language      	= C++,
	basicstyle      = \ttfamily,
	keywordstyle     = \bfseries,
	%identifierstyle = \color{brown},
	commentstyle    = \color{commentgreen},
	otherkeywords	= {self},             % Add keywords here
	emphstyle		= \color{deepred},    % Custom highlighting style
	stringstyle	 	= \color{deepgreen},		
	keywordstyle	=\color{deepblue},
	stringstyle     = \color{magenta},
} % písmo a  barvičky  by možná chtěly doladit - nějaký dobrovolník? 
% volba literate= pro znaky s diakritikou z kódů se zvýrazněnou syntaxí se nesmí používat, rozbíjí překlad

% \begin{lstlisting} Tady je zdrojový kód např. v C++ \end{lstlisting} - pro UTF-8 NE
%\lstinputlisting{source_filename.py} vloží soubor z daného místa a obarví

\usepackage{hyperref} % balíček pro hypertextové odkazy
% \url{www.odkaz.cz}
% \href{http://www.odkaz.cz}{Text který bude jako odkaz}
%\hyperlink{label}{proklikávací_text} - odkaz na text 
% \hypertarget{label}{cíl_odkazu} - cíl odkazu  


\hypersetup{colorlinks=true, linkcolor=myblue, urlcolor=mygreen, citecolor=blue, anchorcolor = magenta,
	linktocpage = true, frenchlinks } % nastavení barvy odkazů 
% bookmarksopen=true, bookmarksnumbered=true, bookmarksopenlevel=1 - nastavuje rozbalování levého menu       





%\usepackage{expl3} % bibtex dependency, must be loaded prior to the bibtex
%\usepackage[backend=bibtex,bibstyle=numeric,sorting=none,date=long,dateabbrev=false,texencoding=utf8,bibencoding=utf8,style=iso-numeric]{biblatex}

\usepackage[a4paper]{geometry}

%\lstset { 
%    language=C++,
%    backgroundcolor=\color{black!5}, % set backgroundcolor
%    basicstyle=\footnotesize,% basic font setting
%}

%\addbibresource{text.bib}
%\nocite{*}

\titlecz{Párty světla} % Název práce
\author{Anna Králová} % Jméno autora
\institution{STŘEDNÍ PRŮMYSLOVÁ A~VYŠŠÍ ODBORNÁ ŠKOLA BRNO, Sokolská 1} % Celý název instituce
\institutiontype{příspěvková organizace} % Typ instituce
\thesistype{Ročníková práce}  % Typ práce/dokumentu
\mentor{Mgr. Miroslav Burda} % Jméno vedoucího práce
%\mentorstatement{Ing. Václava Zavadila} % Jméno vedoucího práce ve čtvrtém pádě

% Newly added
\authorname{Anna}
\authorsurname{Králová}
\schoolyear{2020/2021}
\field{Technické Lyceum} % Studijní obor
\class{L3A}

\placefooter{Brno 2021}

% \usepackage{hyperref} % balíček pro hypertextové odkazy
% \url{www.odkaz.cz}
% \href{http://www.odkaz.cz}{Text který bude jako odkaz}
% \hyperlink{label}{proklikávací_text} - odkaz na text 
% \hypertarget{label}{cíl_odkazu} - cíl odkazu 

%%% Přepínač pracovní kopie
\workcopytrue
%\workcopyfalse

%\renewcommand\bibname{Literatura a zdroje}

\begin{document}
\hyphenation{SOLIDWORKS Solid/-Works}

\newgeometry{margin=2cm, top=3cm, bottom=2.5cm, left=2.5cm, includefoot}

\maketitle

\newgeometry{margin=2cm, top=2.5cm, bottom=2.5cm, left=2.5cm, right=2cm, includefoot}

\makecopyrightstatement{V~Brně}



\pagestyle{empty}

\section*{Zadání}

Cílem této ročníkové práce je sestrojit fukční prototyp malého světýlka za použití programovatelného, inteligentního LED Pásku. 


%\subsection*

\vspace{20mm}

%\section*

%\subsection*

\newpage
\pagestyle{plain}

\tableofcontents % vysází obsah

\setlength{\parskip}{0.4em}

%%% Začátek práce
\setcounter{figure}{0}
\setcounter{table}{0}
\newpage

% Úvod práce
\chapter*{Úvod}
\addcontentsline{toc}{chapter}{Úvod}

 %(viz kapitola \ref{3_kap}).
Každý z nás má rád malé dekorace. Ať už jsou to malé sošky, které pokládáme na poličky v našich pokojích, nebo obrovské vázy, či kusy drúz na ozdobení našich obývacích pokojů. Dekorace ale nemusí být jen takto jednoduchá. Občas se může jednat i o lampičky roztodivných tvarů, nebo vodní mlýnky, které v sobě mají zabudované LED a pomocí cirkulující vody se paprsky světla lámou dopadají na stěnu, což by se dalo nazvat dekorací samo o sobě.
%\cite{schommers}

Mě osobně se vždycky líbily malé průhledné stromečky, které se prodávaly na Vánoce a~jejich barva se postupně měnila. A tak mě napadlo, proč něco takového také nevyrobit, ale tentokrát místo stromečku použít tvar nějaké květiny, která by měnila barvu světla podle ovládání. %(viz kapitola \ref{3_kap}).


Cílem této ročníkové práce je sestrojit prototyp takového dekoračního světla, včetně naprogramování inteligentního LED pásku, dokumentace a možných návrhů, jak tento nápad inovovat.
%\href{https://www.tecomat.cz/products/}{teco}

%todo zadání

%todo odkaz na knihovny

%todo (viz kapitola \ref{3}) --> otázka na učitele?


%Vložíme podle návodu autora programu
%moduluje wifina
%V módech světla -- reference na poslední kapitolu
%Vytvořit na Githubu sloéžku pro původní program, který jsem poté upravila
%přidat subchapters do 4. a 5. kapitoly
%Pájení --> obrázek zapájené soustavy DevKitC a LED 
%Rozepsání se, co je Visual Studio Code Zač
%Propagace letního robotického tábora

%Odkazy? Github?


% Proč?
\begin{figure}[htbp]
	\centering
	\includegraphics[width=0.7\textwidth]{img/4Led2.jpg}
	\caption{Pásek se zapájenými dráty}
	%	\label{fig:install-sdk-3}
\end{figure}\chapter{Použité součástky}
V této kapitole budou představeny a popsány součástky, se kterými budu v této ročníkové práci pracovat.

\section{ESP32-DevkitC}
%Co to je?
ESP32-DevKitC \cite{devkitc-datasheet} je malá programovatelná deska založená na čipu ESP32 od společnosti Espressiff \cite{espressif}. Vstupní a výstupní piny se nachází na obou stranách desky, což umožňuje uživateli připojit periferní zařízení jak pomocí propojovacích vodičů, tak připojením k nepájivému poli. Na desce se také nachází mikro-USB port, který dovoluje  desku jednoduše napájet přímo z počítače, stejně jako nahrávat na desku soubory. \cite{devkitc}

\begin{figure}[htbp]
	\centering
	\includegraphics[width=0.5\textwidth]{img/ESPDevKit3.jpg}
	\caption{ESP32-DevKitC}
	%	\label{fig:install-sdk-3}
\end{figure}

%Co se s tím dá dělat
Deska je kompatibilní s Arduinem \cite{arduino} a často se v kombinaci zrovna s tímto typem desek používá. Díky zabudovanému Wifi modulu, který nese čip ESP32,\cite{ESP32} je možné se na součástku napojit a používat jakékoliv zařízení pracující s Wifi (například chytrý mobil), aby se k součástce napojilo a fungovalo jako dálkový ovladač. V praxi jsem se nejčastěji setkala s použitím mobilu jako dálkového ovladače právě v případě výše zmíněného programování LED světel a řízení drobného robota. S ESP32-DevkitC se dá dělat ale prakticky cokoliv: například se dá použít jako procesor pro po domácku vyrobený alarm, amatérský anemograf na zaznamenávání počasí nebo právě pro  vyrobení malého pojízdného robota na dálkové ovládání. 
%todo Otázka --> udělat čip ESP32 proklikávací, nebo ho nechat v literatuře tak, jak je?

 \href{https://www.espressif.com/sites/default/files/documentation/esp32_datasheet_en.pdf}{ESP32}

%Proč jsem si vybrala zrovna tuto součástku.
Já jsem se rozhodla použít ESP32-DevKitC kvůli jeho dostupnosti, všestrannému využití a taky (možná hlavně) proto, že v mém okolí mělo spoustu lidí s touto technologií zkušenosti a mohli mi v případě nějakého problému jednoduše pomoci. 
Plánovala jsem, že Wifi modul na této desce jednoduše využiji pro dálkové ovládání ESP32-DevKitC z mobilu. Rozměry desky ESP32-DevkitC (přibližně 55 mm na 30 mm) byly také vyhovující pro pozdější teoretické vybudování napájecí \emph{základny}, která měla napájet a řídit LED schované v průhledné plastové květině.
%todo --> Otázka --> může být označení základna kurzívou?


%todo Neopixel distributor
\section{NeoPixel modul s 8 RGB LED WS2812}
%Co to je? Co se s tím dá dělat
Jedná se o pevný pásek s inteligentními LED za sebou. Často se využívá jako výstupní model pro Arduino a obsahuje 8 RGB LED typu WS2812,\cite {WS2812} které lze najít i pod označením NeoPixel\cite{Neopixel}. Výhoda tohoto LED pásku je, že se dá řídit pomocí jednoho datového pinu a dvou napájecích pinů, což umožňuje kontrolér ve WS diodách. Tento modul se sice nehodí našemu původnímu záměru, který vyžaduje aby LED byly na pásku ohebném, ale jako modul pro testování stačí.

%todo Otázka --> V jakém čase mám psát text? Je to vyhovující tak, jak to je?

\begin{figure}[htbp]
	\centering
	\includegraphics[width=0.5\textwidth]{img/NeoPixel2.jpg}
	\caption{NeoPixel modul}
	%	\label{fig:install-sdk-3}
\end{figure}

%Proč jsem si vybrala zrovna tuto součástku.
Tuto součástku jsem se rozhodla použít ze dvou důvodů. Ten první byl, že jsem ještě nevěděla jistě, jaký ohebný LED pásek bych chtěla použít. Druhý důvod byl stejný, jako v případě ESP32. A to, že kdyby nastaly při práci s tímto modulem nějaké problémy, tak jsem znala spoustu lidí, kteří by mi s případnými problémy dokázali pomoci.

\newpage

\section{Pásek WS2812 ohebný}
Po nějaké době práce s předchozím LED páskem jsem si nakonec rozhodla vybrat skoro totožný pásek až na určitou maličkost, a to, aby byl jednoduše ohebný a tvarovatelný. Pracuje a programuje se s ním stejně, jako s předchozím páskem, avšak, jelikož to více vyhovovalo mému záměru, jsem tentokrát použila pásek se čtyřmi RGB LED.\cite{Ohebny}
%todo --> Otázka --> Můžu si dovolit odkaz na amazon? Nemůžu najít oficiálního distributora

\begin{figure}[htbp]
	\centering
	\includegraphics[width=0.5\textwidth]{img/OhebnyLedPasek2.jpg}
%<<<<<<< Updated upstream
	\caption{Pásek WS2812}
%=======
%	\caption{Pásek ws2812}
%>>>>>>> Stashed changes
	%	\label{fig:install-sdk-3}
\end{figure}


\section{Pájení}

Proto, abych s těmito součástkami mohla dál pracovat, jsem potřebovala napojit pevný LED NeoPixel pásek na ESP32-DevKitC. Což jsem udělala tak, že jsem k Neopixel pásku připájela dráty na kontakty: napájení, uzemění a vstupní pin. Na tyto dráty jsem z druhé strany přidělala pinhead a napojila jsem jej z druhé strany na ESP32-DevKitC. Jako vstupní pin jsem zvolila pin č. 21.  

%todo Ukol-J Obrázek zapájené soustavy (Až bude přirozené světlo)

Obrázek: 
Zde jsou zapájené součástky. Po tomto kroku jsem se mohla pustit do programování

%Ostatní poznámky:
%Deska má tři druhy napájení, ale já budu využívat pouze napájení zkrz mikro-USB protože je to pro mě nejjednodušší. Stejně tak do toho budu posílat nahrané soubory z počítače
%Je potřeba vymyslet systém napájení
%Jak to funguje? Jaké je rozložení dané desky? 


% Realizace projektu
\chapter{Program}

\section{V čem psát program a jeho tvorba}

Program jsem se rozhodla psát v programu Visual Studio Code. Pracovala jsem s ním v minulosti a i zde platilo, že jsem kolem měla lidi, který by mi mohli pomoct ve chvíli, když bych měla s něčím problém. Pro psaní kódu jsem použila školní knihovny na programovaní LED světel, které byly původně vyrobeny pro letní Robotický tábor a tak jsem měla o něco lehčí práci, protože jsem knihovny nemusela vytvářet sama.  
Na programování jsou použila pevný LED pásek.


%\href{https://www.tecomat.cz/products/}{teco}

%\lstinputlisting{priklady_c/blikani_LED1.cpp}
\lstinputlisting{Code/Zacatek-programu.cpp}


První dva řádky jsou knihovny, které jsem při psaní využívala. Další tři jsou definované proměnné, které určují, kolik LED světel pásek má (8), jaký pin zajišťuje komunikaci s ESP32-DevKit (21), a číslo, které určuje, jaké číslo ponese první LED světlo s tím,  že další LED byly očíslovány vzestupně.

\newpage

\section{První mód světla}
První barvu prvního módu jsem se rozhodla, že bude červená. Jen obyčejná červená bez jakýkoliv jiných barev nebo blikání. Nejdříve jsem ale musela nadefinovat funkci, která mi pomohla v dalším postupu a usnadnila mi manipulaci se všemi LED. 

%\lstinputlisting{priklady_c/blikani_LED1.cpp}
\lstinputlisting{Code/program1.cpp}


Nejdříve jsem si nadefinovala funkci SetLEDAll, která mi při použití určitého příkazu, jak se LED-ky mají rozsvítit, aplikovala tento příkaz na všechny LED-ky na pásku, a umožnila mi jednodušší manipulaci s programem v pozdější fázi, když jsem všechny tyto módy přepisovala do jednoho programu. 
%Funkce ShowLeds v sobě zase obsahovala dva příkazy, které se v programu nacházeli pro stabilizaci LEDek. 
Na dalším obrázku %todo upravit odkaz
 je ukázán samotný program, který provedl to, že zavolal funkci, že se má daná barva aplikovat na všechny LED-ky a nastavila je na tlumeně červenou barvu. Tlumeně červenou proto, že LEDky zářily opravdu intenzivně a při jasu na 255 mě z nich začaly bolet oči. Proto jsem jas červené nechala na pouhé čtvrtině toho, co by LEDky zvládly.

\section{Druhý mód světla}
%nechci aby to blikalo jako blázen 
Jako druhý mód světla jsem se rozhodla, že použiju světle modrou barvu, která se bude pomalu stupňovat a pak tlumit a vznikne efekt pulzujícího modrého světla. Tohoto výsledku jsem docílila pomocí programu: 

\href{https://www.tecomat.cz/products/}{teco}

%\lstinputlisting{priklady_c/blikani_LED1.cpp}
\lstinputlisting{Code/program2.cpp}

Tento program je navrhnut tak, že se zhasnuté LEDky začnou pomalu rozsvěcovat do jasné modré a poté znovu zhasínat do naprosté tmy. A jelikož byl tento příkaz napsán v části loop, tento cyklus se bude opakovat do té doby, dokud nebude zavolána jiná funkce, nebo nebude ESP32-DevkitC odpojena. Docílí se tak moc pěkného pulzujícího modrého efektu. Opět zde není použita plná síla LED světel, aby byly šetřeny moje oči.  


\section{Třetí mód světla}
Jelikož mi nápad na to, jak jsem udělala pomalu pulzující světlo, připadal skvělý, rozhodla jsem se něco podobného aplikovat i na třetí mód. Tentokrát jsem ale zkusila, jaké by to bylo, kdyby červená barva pomalu přešla do žluté a ze žluté do zelené a pak zase zpět. 

\href{https://www.tecomat.cz/products/}{teco}

%\lstinputlisting{priklady_c/blikani_LED1.cpp}
\lstinputlisting{Code/program3.cpp}

Program na tenhle mód je hodně jednoduchý a využívá podobného principu jako ten předchozí. Opět tu je zpoždění, aby světlo bláznivě neblikalo. Původně jsem chtěla, aby barva připomínala oheň,což mi moc nevyšlo ale i tak to je moc pěkný přechod barvy, který při umístění do plastového krystalu nebo růže (něco jako žárovky) vytvoří nádherný efekt.


\section{Čtvrtý mód světla}
Původně jsem nechtěla mít víc, než tři módy na mých LEDkých, ale neodolala jsem vidině, že budu mít mód, který mi zařídí, že barevný přechod bude procházet celým barevným spektrem  dokola a dokola. 

%nevím co s tímto obrázkem. Jak udělat aby šel text hned vedle obrázku? 
\href{https://www.tecomat.cz/products/}{teco}

%\lstinputlisting{priklady_c/blikani_LED1.cpp}
\lstinputlisting{Code/program4.cpp}


V podstatě na tomto programu není vůbec nic nového. Zase začíná na červené, kdy pomalu projde přes žlutou k zelené, dále pokračuje na světle modrou a tmavě modrou, až se dostane z fialové a z fialové zpátky na začátek, což znamená na červenou. Tenhle efekt vypadá moc pěkně a vážně se pro barevné lampičky hodí. 


%Obrázky kódu? 
%Název školních knihoven nebo nějaká bližší specifikace? Kde se dají sehnat a stáhnout? 
%nevím jakým způsobem napsat zmínění knihoven
%const int CHANNEL = 0;???

\newpage

% Návody na instalaci a rozjetí SolidWorks
\chapter{Ovládání přes mobil}
%Nemám nejmenší tušení co jsem psát, protože o té webovce nic, ale vůůůůůbec nic nevím

\section{Esp32-RBGridUI-Designer} 
Pro vytvoření ovládacího prostředí na mobilu, jsem použila {RBGridUI-Designer}, \cite{RBGridUI-Designer} což je prostředí vyvinuté přímo Robotárnou,\cite{robotarna} pro práci s LED a jejich ovládání přes mobil.

Webová stránka {\em Esp32-RBGridUI-Designer} má toto rozložení: 

\begin{figure}[htbp]
	\centering
	\includegraphics[width=1\textwidth]{img/Esp32-RBGridUI-Designer.png}
	\caption{Prostředí stránky {\em GridUI-Designer}}
	%	\label{fig:install-sdk-3}
\end{figure}

\begin{enumerate}
	\item Postraní lišta, při kliknutí na jakoukoliv komponentu, je uživatel schopen danou kamponentu přetáhnout na manipulační plochu 
	\item Manipulační plocha, na kterou se umisťují komponenty z postranní lišty. Nastavuje se tu jejich poloha a umístění. Manipulační plocha určuje vzhled řídící plochy v aplikaci v mobilu při napojení na daný Wifi modul. 
	\item Soubor Layout, který je potřeba stáhnout a nahrát do stejné složky jako máme hlavní program. Tento soubor nese informace toho, jakým způsobem jsme nanesli komponenty na manipulační plochu.
	\item  {\bf Něco} %todo Otázka --> vysvětlit Co to je?
	\item Bližší specifikace, určující umístění, barvu, a název tlačítka, které bylo vytvořeno na manipulační ploše. Tyto informace je možné zde i předělávat a upravoat 
	\item Tlačítko Reset které vymaže veškeré komponenty z Manipulační plochy a tím pádem i z Layoutu.
\end{enumerate}

%todo Otázka ---> Jak mám upravit rozložení stránky?
{\em Esp32-RBGridUI-Designer} je propojený s aplikací {\em RBControler,} \cite{RBControler}

\section{Vytvoření tlačítek}
Aby my ale LED dělaly přesně to, co chci, musela jsem nastavit na  4 tlačítka
Abych mohla barvy na LED ovládat a libovolně mezi nimi přepínat, rozhodla jsem se v {\em RBGridUI-Designeru} vytvořit 5 tlačítek: co tlačítko, to jeden mód a tlačítko na vypnutí světel. Tlačítka jsem si pojmenovala tak, abych se v barvách vyznala a trochu jsem si i pohrála s barvami tlačítek. Poté mi už jen stačilo zkopírovat soubor layout do samostatného textového souboru, pozměnit příponu souboru z {\em .txt} na {\em .hpp} a umístit ho do stejné složky, jako hlavní program.
%todo Otázka --> obrázek z FreeComanderu?

%\begin{figure}[htbp]
%	\centering
%	\includegraphics[width=0.8\textwidth]{img/Esp32-RBGridUI-Designer - Tlačítka.png}
%	\caption{Tlačítka}
	%	\label{fig:install-sdk-3}
%\end{figure}
%sooooo messy...
\section{Hlavní Program} 
Nyní nastal čas spojit veškeré programy jednotlivých barevných módů do jediného a finálního programu, který budeme používat a bude naším konečným výsledkem po programovací stránce. 
Naštěstí jsem část finální program nemusela tvořit úplně z ničeho a využila jsem části jednoho \href{https://github.com/Nemesis-Rain/Supplements-/tree/main/Originální%20program%20%2B%20layout}{darovaného programu}.
 Tyto části se zabývaly propojením ESP32-DevKitC s mobilem pomocí Wifi modulu. Stačilo tedy barevné módy do programu šikovně zapracovat a vymazat nepotřebné kusy kódu.

%todo Otázka --> Mám problémy s proklikávacími odkazy na github
%todo Ukol_J --> Nahrát Hlavní finální program na Github, stejně jako platform.io 

Následující program ukazuje jen části Hlavního programu. 

Funkce {\em void Setup()} ukazuje nastavení jména a hesla na Wifi, propojení s {\em Layout.hpp} ve stejné složce a pak definování a funkci jednotlivých tlačítek, které jsme si nastavili už předtím v {\em RBGridUI-Designeru.}

Funkce {\em void Loop()} dokončuje ukázku definování tlačítek, která je provedená pomocí příkazu {\em switch.}


\lstinputlisting{Code/new-program - Setup.cpp}

\lstinputlisting{Code/new-program - Loop.cpp}

\newpage

\section{Propojení mobilu a ESP32-DevkitC a kontrola funkčnosti}
Na Google play na mobilu jsem si našla aplikaci jménem \href{https://play.google.com/store/apps/details?id=com.tassadar.rbcontroller}{RB Controler}, také vytvořená pod vedením \href{https://helceletka.cz/robotarna/}{Robotárny}. Aplikaci jsem si nainstalovala a spustila.

Poté jsem zapojila ESP32-DevkitC, ověřila jsem si, že je v něm nahraný Hlavní program a připojila jsem se na wifinu jménem LEDLED. 

Po aktualizaci aplikace se mi na obrazovce objevilo:

%todo Ukol-J --> Zde bude screenshot obrazovky mého mobilu.

Teď už jenom chtělo vyzkoušet, jak moc je daný program kompatibilní a funkční.
 
%todo odkaz na video? +++ VIDEO


\section{Vyměnění LED pásků}

Jelikož mi LED blikali přesně tak, jak jsem zamýšlela, rozhodla jsem se provést poslední krok, a to k výměně pevného LED pásku za Pásek ohebný, který jsem zmínila předtím už v první kapitole. 

Nebylo to nic  těžkého. Tři Krátké dráty jsem připájela na tři předem připravené kontakty: GND , DIN a 5V. Na druhé konce drátů jsem přidělala pinheady, kteřé jsem připojila k ESP32-DevkitC.Jako vstupní pin jsem znovu zvolila pin 21.

%todo Ukol-J --> Sem přjjde obrázek zapájeného ohebného pásku 

Aby tento pásek fungoval, musela jsem v programu pozměnit proměnnou LED_COUNT z 8 na 4, protože ohebný pásek narozdíl od testovacího měl pouze 4 LED.

%todo Ukol-J --> Ukázat zde část programu s LED_COUNT

\newpage


% Modelování základních stroj.  součástí
\chapter{Ověření Funkčnosti a vyměnění LED pásků}




\newpage

% Operace v sestavách
\chapter{Další možné využití}

%Původní nápad -- proč je použit ohebný pásek? 		
Mým původním nápadem bylo, že mojí ročníkovou prací bude výroba plnohodnotné lampičky, počítaje jak elektrotechnickou část tak tu Designovou. Z toho bohužel sešlo z časových důvodů a tak tu teoreticky proberu Designovou část. 
Důvod proč jsem později zaměnila pevní LED pásek za ohebný, byl ten, že původně jsem měla nápad, že ohebný LED pásek by se dal lépe schovat do předem připravené plastové {\em žárovky}. 
{\em Žárovka} by byla tvořená z průhledné Křišťálové pryskyřice tzv. Epoxy resin\cite{epoxyresin}, což je čirá odlévací tekutina používající se jak na umělecké, tak technické účely. Z toho sešlo kvůli problémům zvládnutí této techniky a nedostatku času. 



%Další nápady -- Více světel bez/drátově propojená. 		
Také by bylo možné naprogramovat více ESP32-DevkitC s LED, které by se mohli navzájem propojit, ať už dráty, anebo bezdrátově. V bezdrátové verzi vidím víc výhod, ale zase by mohl být problém s dosahem a dálkou od ovládacího zařízení. 
Na LED pásky by se dalo vyrobit více tzv. {\em žárovek} tím, že by se podle předem vytvořeného originálu vyrobila forma na silikon a ta by se následně opakovaně použila na odlití Epoxidové pryskyřice. Každá ze světel by měla tzv. {\em základnu}, která by byla buď vytvořená z dřevěných desek přesně vypálených na laseru, anebo by se vytiskl plastový obal na 3D tiskárně. Tato možnost se mi asi líbí nejvíc.

	
%todo laser odkaz??? 3d tiskárna odkaz?		

S trochou snahy, kreativity a hodně práce by se z tohoto prototypu mohla stát i sada pro "začínající" programátory a elektroniky, protože to pokryje jak pájení a práci s elektronickými součástkami, tak programování kdy člověk muže buď už předem naprogramovaný program stáhnout z internetu, nebo si ho naprogramovat a upravit podle sebe Obávám se ale že s tímto nápadem by byli menší problémy, protože celkově jak knihovny, tak aplikace, původní program a návod jak s tím pracovat vznikl v rámci Letního Robotického tábora který pořádá každý rok Helceltka, a tím pádem by mohl nastat problém ohledně plagiátorství nápadu 

%todo reference na aplikaci rbcontrol; původní program; Robotický tábor

Ráda bych tuto ročníkovou práci ukončila s myšlenkou, že můj protoryp svítících LED světel má kreativní využití ve všech směrech a je jen na konstruktérovi, jak bude tento nápad chtít využít. 


%   \begin{figure}[htbp]
%	\centering
%	\begin{minipage}[b]{0.5\textwidth}
%		\centering
%		\includegraphics[width=0.75\textwidth]{img/015 img/definovane-funkce.png}
%		\caption{Funkce}
%		%		\label{fig:gear-sketch1}
%	\end{minipage}
%	\qquad
%	\begin{minipage}[b]{0.4\textwidth}
%		\centering
%		\includegraphics[width=1\textwidth]{img/015 img/Program1-červená.png}
%		\caption{Program}
%		%		\label{fig:gear-sketch2}
%	\end{minipage}
%\end{figure}

\newpage

% Návody z výkresové dokumentace
%%%% Výkresovka - textové přepisy návodů
\chapter{Výkresová dokumentace - vybrané návody}
Tato kapitola obsahuje text. přepis dvou návodů na nejběžnější operace v~sestavách.
Text je brán jako doplněk videí vypsaných v~kapitole \ref{videa-sestavy} přílohy \ref{released-videos}.

\section{Popisové pole a uživatelské vlastnosti}
Popisové pole je nedílnou součástí každého výkresu.
Udává údaje o~dané součásti, nebo sestavě, jako je označení, materiál, rozměr a podobně.
V~SolidWorks se vyplňování popisového pole řeší pomocí uživatelských vlastností, které se následně automaticky propisují na výkres.
Pro to, aby tyto vlastnosti správně fungovaly je nutné mít správně nainstalované šablony.
Jejich instalace je popsaná v~návodu \ref{instalace-sablon}.

\begin{figure}[htbp]
    \centering
    \begin{minipage}[b]{0.45\textwidth}
        \centering
        \includegraphics[width=1\textwidth]{img/030 img/vlastnosti-dilu.png}
        \caption{Tabulka vlastností souč.}
        \label{fig:realview-1}
    \end{minipage}
    \qquad
    \begin{minipage}[b]{0.45\textwidth}
        \centering
        \includegraphics[width=0.7\textwidth]{img/030 img/upravit-material.png}
        \caption{Tl. pro změnu materiálu}
        \label{fig:realview-2}
    \end{minipage}
\end{figure}

Otevřeme si díl, kterému chceme upravit vlastnosti.
V~nabídce \B{Soubor} vybereme možnost \B{Vlastnosti}.
Zobrazí se nabídka vlastností dílu, ve které se musíme přepnout na kartu \B{Závislý na konfiguraci}.
Ve sloupci \B{Hodnota/textový výraz} můžeme měnit položky popisového pole.
Jakmile podle potřeby nastavíme všechny hodnoty, uložíme změny stiskem tlačítka \B{OK}.
Pro změnu kolonky \B{Materiál} musíme nastavit materiál ve Stromu FeatureManageru dané součásti.

\section{Drážka pro pero na hřídeli}
Při tvorbě výkresové dokumentace hřídele se často nevyhneme popisování drážky pro pero.
Začneme tím, že na výkres vložíme pohled hřídele tak, abychom viděli celou drážku pro pero (viz \autoref{fig:keyslot-dwg} vlevo).
Do pohledu nesmíme zapomenout přidat osu.
Na kartě \B{Výkres} vybereme \B{Řez}.
\begin{figure}[htbp]
    \centering
    \includegraphics[width=0.65\textwidth]{img/030 img/perodrazka-hridel-screen.png}
    \caption{Popis drážky pro pero na hřídeli}
    \label{fig:keyslot-dwg}
\end{figure}

Vlevo v~nabídce nastavení řezu zvolíme svislou orientaci a řeznou čáru umístíme přibližně do středu drážky.
Zkontrolujeme si, že řez směřuje ven od středu hřídele.
Do zobrazení řezu umístíme středovou značku.

Když máme pohledy nachystané, můžeme začít popisovat.
V~pohledu zakótujeme délku drážky s~podržením klávesy \It{SHIFT} a výběrem obou krajních oblouků.
K~této kótě přidáme oboustrannou toleranci +0,3 a -0.
Dále zakótujeme vzdálenost drážky (krajního oblouku) od čela hřídele, nebo nejbližšího osazení.

Přejdeme do řezu, kde nejdříve zakótujeme šířku drážky.
Této kótě přidáme toleranci \B{P9}.
Dále zakótujeme hodnotu zaoblení dna drážky.
Posledním kótovaným rozměrem je hloubka drážky, kterou zadáme vůči protilehlému oblouku, viz \autoref{fig:keyslot-dwg} vlevo.
Zde přidáme toleranci hloubky, která je pro každou skupinu rozměrů drážek jiná -- zjistíme ji z~tabulky \ref{tab:pera-tesna}.

Posledním krokem je přidání značek drsností povrchu.
U~boků drážky se jedná o~drsnost Ra 3,2 -- značku umístíme na kótu udávající šířku drážky, viz \autoref{fig:keyslot-dwg}  vpravo.
Povrch dna drážky bude mít drsnost Ra 6,3 a umístíme jej na kótu udávající hloubku, opět viz \autoref{fig:keyslot-dwg}  vpravo.

\section{Drážka pro pero v~náboji}
Na výkres si umístíme přední pohled na náš náboj (např. ozubené kolo).
V~kartě \B{Popis} klikneme na \B{Detail}.
Střed detailního pohledu umístíme do středu díry v~náboji a jeho velikost nastavíme tak, aby byla celá drážka viditelná.
Nesmíme zapomenout přidat středové značky.
Vzhledem k~tomu, že díra s~drážkou prochází skrz náboj, její délku kótovat nemusíme.

V~zobrazení detailního pohledu ale potřebujeme zaznačit průměr díry, šířku drážky, její hloubku, velikost zaoblení, drsnosti povrchu a odpovídající tolerance.
Začneme průměrem.
Podržíme klávesu \It{SHIFT} a klikneme nejdříve na první a následně druhou stranu oblouku -- vytvoříme tak průměrovou kótu.
K~ní ještě doplníme toleranci.
Vzhledem k~tomu, že se jedná o~díru, můžeme zvolit například toleranci H7 (možné další viz \autoref{fig:jednotna-dira}).

\begin{figure}[htbp]
    \centering
    \includegraphics[width=0.6\textwidth]{img/030 img/perodrazka-naboj-screen.png}
    \caption{Plně popsaná drážka na pero v~náboji}
    \label{fig:keyhole-dwg}
\end{figure}

Po průměru musíme zakótovat šířku drážky.
Tato kóta bude mít toleranci P9, jako jsme u~drážek na pero zvyklí.
Hloubku drážky označíme stejně jako u~protikusu na hřídeli -- klikneme na hranu dna drážky, podržíme \It{SHIFT} a klikneme na protilehlý oblouk.
Ještě přidáme toleranci pro danou velikost pera -- zjistíme z~tabulky \ref{tab:pera-tesna}.
Zakótujeme zaoblení a přidáme značky drsnosti povrchu viz \autoref{fig:keyhole-dwg}.

\newpage

% Zaver prace
\chapter{Závěr}
Prototyp světla je ke dni odevzdání ročníkové práce plně funkční. Do budoucna by se dal zkonstruovat obal, ve kterém by byla uložená elektronika (hlavně ESP32-DevKitC) a průhledný tvar, skrz který by LED zářily. Dále by se dalo vyrobit a zprovoznit těchto světel víc a zařídit, aby každé z nich bylo propojeno přes wifi s mobilním telefonem.


Tato ročníková práce mi dala hodně zabrat. Ne proto, že bych měla složité téma a nebo program by byl složitý na naprogramování, ale problém byl v tom, že jsem se musela často učit pracovat s programy a věcmi, se kterými jsem předtím nepracovala. Naučila jsem se pracovat s ESP32 a inteligentními LED světly, naučila jsem se je programovat a dokonce se napojit na ESP32-DevkitC mobilním telefonem a pomocí něj ESP32-DevkitC ovládat.
Celkově mě ale práce na této ročníkové práci bavila a dala mi do budoucna spoustu zkušeností, které se jistě budou hodit. 



\clearpage
\phantomsection

%\appendix
%\addcontentsline{toc}{chapter}{Přílohy}

% Prilohy
%\chapter{Seznam již vydaných videí} \label{released-videos}
Tato příloha obsahuje kompletní seznam videí vzniklých v~rámci projektu P3D vč. odkazů rozdělených dle jednotlivých témat. \newline
\noindent\It{Pozn.: při kliknutí na odkaz budete přesměrování na stránku korespondujícího videa (pouze v~digitální verzi)}.

\section{Instalace a zprovoznění SolidWorks SDK} \label{videa-instalace}
\href{https://aka.parallaxproduction.cz/instalaceSDK}{Instalace a první spuštění SolidWorks SDK 2020/2021 (aka.parallaxproduction.cz/instSDK)} \newline
\href{https://aka.parallaxproduction.cz/sablony}{Instalace šablon a knihoven norm. dílů ze Sokolské (aka.parallaxproduction.cz/sablony)} \newline
\href{https://aka.parallaxproduction.cz/realview}{Aktivace Realview na necertifikované grafické kartě (aka.parallaxproduction.cz/realview)} \newline

\section{Základy modelování} \label{videa-modelovani}
\href{https://aka.parallaxproduction.cz/jednoducha-pruzina}{Jednoduchá pružina (aka.parallaxproduction.cz/jednoducha-pruzina)} \newline
\href{https://aka.parallaxproduction.cz/j-ozubene-kolo}{Ozubené kolo s~přímým čelním ozubením (aka.parallaxproduction.cz/j-ozubene-kolo)} \newline
\href{https://aka.parallaxproduction.cz/vyk-oz-kolo}{Ozubené kolo pro výkres - obálka (aka.parallaxproduction.cz/vyk-oz-kolo)} \newline
\href{https://aka.parallaxproduction.cz/jednorad-r-kolo}{Jednořadé řetězové kolo (aka.parallaxproduction.cz/jednorad-r-kolo)} \newline
\href{https://aka.parallaxproduction.cz/perodrazka-naboj}{Drážka pro pero v~náboji (aka.parallaxproduction.cz/perodrazka-naboj)} \newline
\href{https://aka.parallaxproduction.cz/perodrazka-hridel}{Drážka pro pero na hřídeli (aka.parallaxproduction.cz/perodrazka-hridel)} \newline

\section{Výkresová dokumentace} \label{videa-vykresy}
\href{https://aka.parallaxproduction.cz/popisove-pole}{Popisové pole a už. vlastnosti na výkrese (aka.parallaxproduction.cz/popisove-pole)} \newline
\href{https://aka.parallaxproduction.cz/vykres-perodrazka-na}{Výkres drážky pro pero v~náboji (aka.parallaxproduction.cz/vykres-perodrazka-hr)} \newline
\href{https://aka.parallaxproduction.cz/vykres-perodrazka-hr}{Výkres drážky pro pero na hřídeli (aka.parallaxproduction.cz/vykres-perodrazka-hr)} \newline

\section{Práce se sestavami} \label{videa-sestavy}
\href{https://aka.parallaxproduction.cz/prejmenovani-dilu}{Přejmenování dílu v~sestavě (aka.parallaxproduction.cz/prejmenovani-dilu)} \newline
\href{https://aka.parallaxproduction.cz/pack-and-go}{Přesun sestavy pomocí Pack and Go... (aka.parallaxproduction.cz/pack-and-go)} \newline

\chapter{Seznam plánovaných témat} \label{planned-videos}
V~následujícím seznamu naleznete další témata, pro která jsou videa plánována.

\subsection*{Nastavení a úpravy SolidWorks}
\begin{itemize}
    \setlength\itemsep{0.05em}
    \item Upgrade SolidWorks na novou verzi (např. 2020 na 2021),
    \item vlastní úpravy uživatelského prostředí,
    \item doplňkové moduly SolidWorks.
\end{itemize}

\subsection*{Základy modelování}
\begin{itemize}
    \setlength\itemsep{0.05em}
    \item Řetěz,
    \item řemen,
    \item plechové díly (série),
    \item svařované konstrukce (série),
    \item další normalizované prvky.
\end{itemize}

\subsection*{Výkresová dokumentace}
\begin{itemize}
    \setlength\itemsep{0.05em}
    \item Kusovník na výkrese sestavy,
    \item 
\end{itemize}

\subsection*{Sestavy}
\begin{itemize}
    \setlength\itemsep{0.05em}
    \item Základní, upřesňující a strojní vazby (série),
    \item konfigurace (pravděpodobně série).
\end{itemize}

%\chapter{Obrazové přílohy}

%\begin{figure}[h]
%    \centering
%    \includegraphics[width=0.85\textwidth]{img/ToBeRemoved/PPSB-T_BOTH.png}
%    \caption{Vizualizace PPSB-T (horní strana vpravo, dolní vlevo).}
%    \label{fig:PPSB-T_VISUAL}
%\end{figure}

\chapter{Vybrané normy}

\begin{table}[] \catcode`\-=12
    \centering
    \begin{tabular}{c|cc|c|c|c|ccccc}
    \multirow{2}{*}{\B{Průměr D}} & \multicolumn{3}{c|}{\B{Rozměr drážky}}      & \multicolumn{2}{c|}{\B{Mezní úchylky hloubky}}                                                                                              & \multicolumn{5}{c}{\B{Rozměry pera}}                                  \\ \cline{2-11} 
                                  & t   & t\subscript{1} & R\subscript{1}       & \multicolumn{1}{l|}{na hřídeli (t)}                                  & \multicolumn{1}{l|}{v náboji (t\subscript{1})}                       & b  & h  & R                     & l\subscript{min} & l\subscript{max} \\ \hline
    6 až 8                        & 1,1 & 0,9            & \multirow{2}{*}{0,2} & \multirow{4}{*}{\begin{tabular}[c]{@{}c@{}}+0,1\\ -0,0\end{tabular}} & \multirow{9}{*}{\begin{tabular}[c]{@{}c@{}}+0,2\\ +0,1\end{tabular}} & 2  & 2  & \multirow{2}{*}{0,25} & 9                & 20               \\
    8 až 10                       & 1,7 & 1,3            &                      &                                                                      &                                                                      & 3  & 3  &                       & 9                & 36               \\ \cline{1-4} \cline{7-11} 
    10 až 12                      & 2,4 & 1,6            & \multirow{4}{*}{0,4} &                                                                      &                                                                      & 4  & 4  & \multirow{4}{*}{0,5}  & 10               & 45               \\
    12 až 17                      & 2,9 & 2,1            &                      &                                                                      &                                                                      & 5  & 5  &                       & 12               & 56               \\ \cline{5-5}
    17 až 22                      & 3,5 & 2,5            &                      & \multirow{8}{*}{\begin{tabular}[c]{@{}c@{}}+0,2\\ -0,0\end{tabular}} &                                                                      & 6  & 6  &                       & 16               & 70               \\
    22 až 30                      & 4,1 & 2,9            &                      &                                                                      &                                                                      & 8  & 7  &                       & 20               & 90               \\ \cline{1-4} \cline{7-11} 
    30 až 38                      & 4,7 & 3,3            & \multirow{6}{*}{0,6} &                                                                      &                                                                      & 10 & 8  & \multirow{6}{*}{0,7}  & 25               & 110              \\
    38 až 44                      & 4,9 & 3,1            &                      &                                                                      &                                                                      & 12 & 8  &                       & 32               & 140              \\
    44 až 50                      & 5,5 & 3,5            &                      &                                                                      &                                                                      & 14 & 9  &                       & 40               & 180              \\ \cline{6-6}
    50 až 58                      & 6,2 & 3,8            &                      &                                                                      & \multirow{3}{*}{\begin{tabular}[c]{@{}c@{}}+0,4\\ +0,2\end{tabular}} & 16 & 10 &                       & 45               & 200              \\
    58 až 65                      & 6,8 & 4,2            &                      &                                                                      &                                                                      & 18 & 11 &                       & 50               & 220              \\
    65 až 75                      & 7,4 & 4,6            &                      &                                                                      &                                                                      & 20 & 12 &                       & 56               & 250             
    \end{tabular}
    \caption{Výběr z~normy ČSN 02 2562 - Pera těsná\cite{ST}}
    \label{tab:pera-tesna}
\end{table}

\begin{figure}[htbp]
    \centering
    \includegraphics[width=1\textwidth]{img/jednotna-dira.JPG}
    \caption{Soustava jednotné díry, tučně zvýrazněné hodnoty jsou doporučené, převzato z~\cite{ELUC-DIRA}}
    \label{fig:jednotna-dira}
\end{figure}

\begin{figure}[htbp]
    \centering
    \includegraphics[width=1\textwidth]{img/jednotna-hridel.JPG}
    \caption{Soustava jednotného hřídele, tučně zvýrazněné hodnoty jsou doporučené, převzato z~\cite{ELUC-HRIDEL}}
    \label{fig:jednotna-hridel}
\end{figure}

\input literatura.tex


\listoffigures
\addcontentsline{toc}{chapter}{Seznam obrázků}

%\listoftables
%\addcontentsline{toc}{chapter}{Seznam tabulek}

\end{document}
