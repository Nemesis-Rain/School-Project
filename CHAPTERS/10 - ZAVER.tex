\chapter{Závěr}
Cílem mé dlouhodobé maturitní práce bylo

Nejdříve jsem sestavila obvod, který řídila deska ESP32-DevkitC. Tento obvod se zapínal a vypínal jediným přepínačem, který přesušoval a spojoval obvod. Součástí obvodu je Kromě li-ion baterií a regulátor napětí, které napětí baterií sráží na požadovanou hodnotu. Dodatečně jsem k bateriím vytvořila ještě nabíječku, kterou se kdykoliv baterie mohly znovu nabít. 

Navrhla jsem si v aplikaci ovládací panel a tlačítky a naprogramovala ESP32-DevkitC tak, aby se pomocí jeho WiFi modulu daly jednotlivé módy světla z mobilního zařízení bezdrátově ovládat.  

Dále jsem si vytvořila silikonovou formu, do které jsem opakovaně odlila z křišťálové pryskyřice 3 růže, abych je mohla později použít jako žárovky na jednotlivé prototypy. 

Jako poslední jsem navrhla Základnu, do které se vložil Elektrický obvod, k vrchnímu dílu se připojila růže, která sloužila jako lampička a skrz kterou LED pásek prosvěcoval do prostoru.






%Prototyp světla je ke dni odevzdání ročníkové práce plně funkční. Do budoucna by se dal zkonstruovat obal, ve kterém by byla uložená elektronika (hlavně ESP32-DevKitC) a průhledný tvar, skrz který by LED zářily. Dále by se dalo vyrobit a zprovoznit těchto světel víc a zařídit, aby každé z nich bylo propojeno přes wifi s mobilním telefonem.


%Tato ročníková práce mi dala hodně zabrat. Ne proto, že bych měla složité téma a nebo program by byl složitý na naprogramování, ale problém byl v tom, že jsem se musela často učit pracovat s programy a věcmi, se kterými jsem předtím nepracovala. Naučila jsem se pracovat s ESP32 a inteligentními LED světly, naučila jsem se je programovat a dokonce se napojit na ESP32-DevkitC mobilním telefonem a pomocí něj ESP32-DevkitC ovládat.
%Celkově mě ale práce na této ročníkové práci bavila a dala mi do budoucna spoustu zkušeností, které se jistě budou hodit. 

 

