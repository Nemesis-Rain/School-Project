\chapter*{Úvod}
\addcontentsline{toc}{chapter}{Úvod}


Každý z nás má rád malé dekorace. Ať už jsou to malé sošky, které pokládáme na poličky v~našich pokojích, nebo obrovské vázy, či kusy drahých kamenů na ozdobení našich obývacích pokojů. Dekorace ale nemusí být jen takto jednoduchá. Občas se může jednat i o lampičky různých tvarů, nebo vodní mlýnky, které v sobě mají zabudované LED a pomocí cirkulující vody se paprsky světla lámou dopadají na stěnu, což by se dalo nazvat dekorací samo o sobě.


Mě osobně se vždycky líbily malé průhledné stromečky, které se prodávaly na Vánoce a~jejich barva se postupně měnila. A tak mě napadlo, proč něco takového také nevyrobit, ale tentokrát místo stromečku použít tvar nějaké květiny, která by měnila barvu světla podle ovládání. 


Pro realizaci této dlouhodobé maturitní práce bych z důvodu jednoduché manipulace chtěla použít Desku ESP32-DevKitC s připojeným LED páskem. Obě komponenty by mohly být napájeny bateriemi a v obvodu by nesmělo chybět přepínací tlačítko, které bude celý obvod zapínat a vypínat.

Poté bych chtěla do ESP32-DevKitC naprogramovat různé módy světel a využít jeho Wifi modulu, abych se k desce mohla připojit mobilním telefonem a LED pásek bezdrátově ovládat. 

Dále bych chtěla vyrobit silikonovou formu na květ růže, a odlít do ní květ z křišťálové pryskyřice, která krásně imituje sklo.

Pak už by jen stačilo navrhnout a vyrobit nějakou krabičku, na jejíž vršek se umístí odlitá květina a celý obvod se schová do dřevěné krabičky. Ráda bych vytvořila tři prototypy těchto světel.

Cílem mé dlouhodobé maturitní práce je...
