\chapter*{Úvod}
\addcontentsline{toc}{chapter}{Úvod}

Každý z nás má rád malé dekorace. Ať už jsou to malé sošky, které pokládáme na poličky v našich pokojích, nebo obrovské vázy nebo kusy drůz na ozdobení našich obývacích pokojů. Dekorace nemusí být ale jen takto nudné. Občas to bývají lampičky roztodivných tvarů, nebo vodní mlýnky, které v sobě mohou mít zabudovaná nějaká světla, která by se v cirkulující vodě nádherně leskla.
%\cite{schommers}

Mě osobně se vždycky líbily malé průhledné stromečky, které se prodávaly na vánoce a jejich barva se postupně měnila. A tak mě napadlo, proč něco takového také nevyrobit, ale tentokrát místo stromečku použít tvar nějaké květiny, která by měnila barvu světla podle ovládání. %(viz kapitola \ref{3_kap}).

Cílem této ročníkové práce je sestrojit prototyp takového dekoračního světla, včetně naprogramování inteligentního Led pásku, dokumentace a možných návrhů, jak tento nápad inovovat.
%\href{https://www.tecomat.cz/products/}{teco}

