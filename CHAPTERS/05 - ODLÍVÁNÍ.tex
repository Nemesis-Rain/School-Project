\chapter{Výroba růže}
Původním záměrem bylo vytvořit svíticí lampičku ve tvaru růže. Vzorem pro výrobu růže byl plastový vršek od lahvičky s vůní, s rozměry: přibližně 6 cm široká a 3 cm vysoká. Tato růže byla použita na výrobu formy. Forma byla později opakovaně použita na výrobu tří téměř identických odlitků
%Mohla jsem se domluvit s umělcem, aby růže byla ze skla, mohla jsem ji vytisknout na 3D tiskárně. Já jsem se ji ale rozhodla odlít, protože mě zaujala práce s křišťálovou pryskyřicí. 

*Obrázek k vzoru*


\section{Silikonová forma}

Bylo rozhodnuto, že forma bude vyrobena z odlévacího silikonu. Odlévací silikony mají skvělé kopírovací vlastnosti, jednoduchým se odformováním a dostatečnou pružností, potřebnou k dobrému odformování komplikovaných výrobků a pokud jsou dobře udělané, lze je používat opakovaně



\begin{figure}[htbp]
	\centering
	\includegraphics[width=0.5
	\textwidth]{img/05 odl/Silicone mold.jpg}
	\caption{Silikonová forma}
	%	\label{fig:install-sdk-3}
\end{figure}
 

%\subsection{Leukopren žlutý silikon}

\subsection{Adiční silikon GMS A30 (Silikon)}

%Druhým druhem silikonu byl Adiční silikon GMS A30.
Na výrobu formy jsem byl použit Adiční silikon GMS A30, z internetového obchodu Levnetmely.cz. Jedná se o dvousložkový, dobře tekutý silikon s velmi dobrou kopírovací schopností, schopný zkopírovat jak povrch obyčejného papíru, tak dřeva s letokruhy.

Tento silikon se používá na výrobu forem, například pro odlívání epoxidové pryskyřice, sádry nebo vosku. Dále se využívá na výrobu těsnění nebo zalévání součástek v elektroprůmyslu.


\subsection{Práce se silikonem}
Balení Adičního silikonu GMS A30 se skládá ze dvou složek: Složky A a Složky B. Tyto dvě složky se musí smíchat v hmotnostním poměru 1:1. Při míchání je potřeba dát pozor, aby se do Silikonu nedostali bublinky, které by mohli znehodnotit formu a zdeformovat kopírovaný tvar.

Obě složky je potřeba dobře promíchat, ale zároveň je třeba dát pozor, aby se při míchání do silikonu nedostali bublinky, které by mohli znehodnotit formu a zdeformovat tvar, kopírovaného předmětu. V případě použití kompresoru, by stačilo veškerý vzduch odsát a tím způsobem se zbavit bublinek. Pokud ale člověk kompresor nevlastní, malému množství bublinek nevyhne. Při práci se silikonem je potřeba dbát na bezpečnost práce a při manipulaci se silikonem mít na sobě ochranné brýle a rukavice.

*Obrázek formy*

Výsledkem bývá středně tvrdá, odolná silikonová pryžová forma, se skvělým odformováním, které v našem případě trochu problematizoval komplikovaný tvar květiny, 


\begin{figure}[htbp]
	\centering
	\includegraphics[width=0.5
	\textwidth]{img/05 odl/Rose.jpg}
	\caption{Odlitá růže}
	%	\label{fig:install-sdk-3}
\end{figure}

\section{Pryskyřice}
Křišťálová pryskyřice (Epoxy Resin) je dvousložková průhledná odlévací směs často používaná na výrobu bižuterie, různých dekorací nebo glazování povrchů. Dobře se s ní pracuje, její zápach je nízký, a po vytvrdnutí je její povrch lesklý a nesmršťuje se. 

Pro realizaci tohoto projektu byla pryskyřice vybrána z důvodu dobré imitace skla, a zároveň nízké křehkosti tohoto materiálu. 
%Tato pryskyřice byla vybrána i z důvodu, že dobře imituje sklo, ale zase postrádá jeho křehkost.
Při odlévání růže byla konkrétně použita Křišťálová pryskyřice od firmy Gedeo, sehnatelná v jakémkoliv specializovaném uměleckém obchodě nebo obchodě pro kutili. 


\subsection{Práce s pryskyřicí}

Pro použití pryskyřice je třeba smíchat složku A a složku B smíchají v hmotnostním poměru 2:1. Při odvažování je důležitá přesnost a dobře promíchat, jinak by mohlo dojít ke špatnému vytvrzení výrobku. Při míchání je třeba dávat pozor, aby se do pryskyřice nedostali bublinky. Poté se daná směs vylije do předem připravené formy, nebo na povrch, který chceme glazovat a počkat 24 až 48 hodin, než pryskyřice zatvrdne. Opět, Křišťálová pryskyřice je chemikálie a proto se, z důvodů možných alergických reakcí a podráždění dýchacích cest, doporučuje při manipulaci s pryskyřicí, nosit ochranné brýle, rukavice a respirátor. 

Před vylitím pryskyřice do předem připravené formy je prý vhodné danou formu vytřít speciální vazelínou, která prodlužuje životnost formy a která brání jejímu poškození při vyndávání odlitku. Tento způsob se ale, v našem případě, při odlévání pryskyřičné růže, neosvědčil. Vazelína se smíchala s odlévací pryskyřicí a ve výrobku se vytvořil bílý mléčný kal, který znehodnotil výrobek a pokazil sklovitý dojem pryskyřice.







